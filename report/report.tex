\documentclass[journal,transmag]{IEEEtran}
    \hyphenation{op-tical net-works semi-conduc-tor}
    \begin{document}
    \title{ Analysis and Comparison of Brute-Force and Genetic Algorithm methods
        on Traveling Salesman Problem }

    \author{ \IEEEauthorblockN{Onur Temizkan} \IEEEauthorblockA{ Department of
        Computer Engineering, Izmir Institute of Technology, Urla / Izmir /
        Turkey } }

    \markboth{Optimization Methods Term Project, Onur Temizkan, May 2018}{}%

    \IEEEtitleabstractindextext{%
    \begin{abstract}
        \textit{ Abstract: Traveling Salesman Problem (TSP) is a common
            NP-Complete problem in Computer Science and Optimization. Thus,
            there is no known way to solve it in polynomial time at the time of
            this writing. In this report, we will analyse Genetic algorithm
            approach to TSP and compare it with Brute-Force Search method. }
    \end{abstract}

    \begin{IEEEkeywords}
    traveling salesman problem, optimization, genetic algorithm, TSP,
    combinatorics.
    \end{IEEEkeywords}}

    \maketitle
    \IEEEdisplaynontitleabstractindextext \IEEEpeerreviewmaketitle

    \section{Introduction}

    \IEEEPARstart{I}{n} this report, we are going to analyze the performance of
    a genetic algorithm implementation, onto Traveling Salesman Problem to find
    the optimum path to traverse a multi-node undirected and weighted graph. On
    the other side, to measure the pros / cons of the GA approach, we will also
    apply brute-force method on the same problem.

    \section{Problem Definition}

    For this research, we will define the problem as a \textit{undirected graph}
    with N nodes. There are edges with several weights between those nodes. In
    TSP analogy, the nodes represent cities, and edges represent paths between
    those cities. Edge weights, represent the distances between two cities. A
    route is valid if the salesman start traveling from a given node, visit all
    cities once, and return back to the starting city.

    Since the graph is not necessarily a \textit{complete graph}, the algorithm
    should check next available nodes from the node at a single time to create
    the route. All nodes should be visited once. So we should be aware of the
    possible circular routes and should avoid them.

    \section{Methods}
    In this paper, we will use two different methods to the same problem.

    \subsection{Brute-Force Search}
    Brute-Force Search solves this problem in an inefficient but a definite way.
    To find the best route, Brute-Force Search checks all possible path
    permutations without any specific condition or restriction. While checking
    routes, it stores and updates the shortest route from the start of the
    execution. When the checks are finished, the shortest route is the last
    route stored. Since it checks all possible routes, it makes sure that the
    returned result is the best possible solution.

    \subsection{Genetic Algorithm}
    Solving the problem with a Genetic Algorithm is more efficient but
    indefinite. This is because in a GA, we do not necessarily check all
    possible routes, instead we try to evolve a set of routes to create a better
    route. GA approach is an iterative method which is based on the natural
    selection mechanism of evolution. Since natural selection \textit{-thus
    evolution} doesn't have a stopping point, Genetic Algorithms do not have a
    strict finishing point. So we presume that there can always be a better
    solution at the any state of the algorithm execution, and will choose a
    stopping condition.


    \section{Implementation}
    The code \cite{code_repository} of both Brute-Force and Genetic Algorithm
    approaches are implemented in Python. We tried to keep the code simple by
    removing any nonessential parts. This also helped to measure and compare the
    performances since there are almost no avertible computational cost in any
    of those two implementations.

    Since our test example is an incomplete undirected graph, our implementation
    is based on specifically for that kind of graphs, another implementation for
    complete graphs can be also found \cite{code_repository}.

    \section{Results}
    (TBD)

    \section{Conclusion}
    Genetic Algorithm approach works very well on Traveling Salesman Problem.
    That approach significantly reduced the number of iterations to find the
    optimal route, comparing with the Brute-Force Search. Although, our sample
    problem is not so big, it reduced the execution time in a significant
    amount. When the problem gets bigger, the execution time benefits should be
    much more significant.

    \appendices
    \section{Summary of Genetic Algorithms}

    \ifCLASSOPTIONcaptionsoff
      \newpage
    \fi

    \begin{thebibliography}{1}

    \bibitem{code_repository}
    http://github.com/onurtemizkan/tsm-genetic-algorithm
    \end{thebibliography}

    \end{document}
